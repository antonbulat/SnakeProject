%%%%%%%%%%%%%%%%%%%%%%%%%%%%%%%%%%%%%%%%%
% University/School Laboratory Report
% LaTeX Template
% Version 3.1 (25/3/14)
%
% This template has been downloaded from:
% http://www.LaTeXTemplates.com
%
% Original author:
% Linux and Unix Users Group at Virginia Tech Wiki 
% (https://vtluug.org/wiki/Example_LaTeX_chem_lab_report)
%
% License:
% CC BY-NC-SA 3.0 (http://creativecommons.org/licenses/by-nc-sa/3.0/)
%
%%%%%%%%%%%%%%%%%%%%%%%%%%%%%%%%%%%%%%%%%

%----------------------------------------------------------------------------------------
%	PACKAGES AND DOCUMENT CONFIGURATIONS
%----------------------------------------------------------------------------------------

\documentclass{article}

%\usepackage{algorithm}
%\usepackage[noend]{algpseudocode}


%\usepackage[version=3]{mhchem} % Package for chemical equation typesetting
%\usepackage{siunitx} % Provides the \SI{}{} and \si{} command for typesetting SI units
\usepackage{graphicx} % Required for the inclusion of images
\usepackage{natbib} % Required to change bibliography style to APA
\usepackage{amsmath} % Required for some math elements 

\usepackage{amssymb}

\usepackage{listings}
\lstset{numbers=left, numberstyle=\tiny, numbersep=5pt}
\lstset{language=bash}

\setlength\parindent{10pt} % Removes all indentation from paragraphs

\renewcommand{\labelenumi}{\alph{enumi}.} % Make numbering in the enumerate environment by letter rather than number (e.g. section 6)

%\usepackage{times} % Uncomment to use the Times New Roman font

%----------------------------------------------------------------------------------------
%	DOCUMENT INFORMATION
%----------------------------------------------------------------------------------------

%\title{\textsc{Big Data} \\ \textsc{Sheet 05} } % Title
\title{Big Data \"Ubungsblatt 05}
\author{Anton Bulat, Josephine Geiger, Julia Siekiera} % Author name


\date{\today} % Date for the report

\begin{document}

\maketitle % Insert the title, author and date

\section*{Aufgabe 1: Monoide}

\subsection*{a)}
Handelt es sich im Folgenden um Monoide?\\
Ein Monoid ist eine Halbgruppe mit Einselement, also ein Tripel $(G, \circ, e)$ aus:
\begin{enumerate}
\item[-] Menge $G$
\item[-] assoziativer Verkn\"upfung $\circ: G \times G \rightarrow G$
\item[-] Einselement $e \in G$
\end{enumerate}
\subsubsection*{i)}
Das Kreuzprodukt von Vektoren aus $\mathbb{R}^3$ ist kein Monoid, da die Verkn\"upfung des Kreuzproduktes nicht assoziativ ist.\\
Gegenbeispiel: Sei $e_i$ der $i$-te Einheitsvektor.
$$(e_1 \times e_1) \times e_2 = 0, \qquad e_1 \times(e_1 \times e_2) = -e_2.$$
\subsubsection*{ii)}
Die Addition von Matrizen aus $\mathbb{R}^{m\times n}$ mit $m \neq n$ ist ein Monoid mit dem Tripel $(\mathbb{R}^{m \times n}, +, 0_{mn})$, wobei $+$ die elementweise Addition ist und damit assoziativ und $0_{mn}$ die Nullmatrix ist. Addiert man zwei Elemente aus $\mathbb{R}^{m\times n}$, so ergibt sich ein Element, das ebenfalls in $\mathbb{R}^{m\times n}$ liegt.
\subsubsection*{iii)}
Die Bestimmung des gr\"o\ss ten gemeinsamen Teilers ganzer Zahlen ist ein Monoid mit dem Tripel $(\mathbb{Z}, ggT, 0)$, wobei die Verkn\"upfung ggT die Bestimmung des gr\"o\ss ten gemeinsamen Teilers sein soll mit dem geltenden Assoziativgesetz: $$ggT(a,b,c) = ggT(ggT(a,b),c) = ggT(a,ggT(b,c)).$$
Die $0$ ist hier das Einselement, da der $ggT(a,0)=a\quad \forall a \in \mathbb{Z}$.
\subsubsection*{iv)}
Die bin\"are XOR-Verkn\"upfung Boolscher Werte ist ein Monoid mit dem Tripel $(\{False, True\}, XOR, False)$, wobei die XOR-Verkn\"upfung assoziativ ist wie in der Tabelle gezeigt (mit False $\hat{= }\ 0$, True $\hat{= }\ 1$):
$$
\begin{tabular}{ccccc}
A & B & C & (A XOR B) XOR C & A XOR (B XOR C)\\
0 & 0 & 0 & 0 & 0\\
0 & 0 & 1 & 1 & 1\\
0 & 1 & 0 & 1 & 1\\
0 & 1 & 1 & 0 & 0\\
1 & 0 & 0 & 1 & 1\\
1 & 0 & 1 & 0 & 0\\
1 & 1 & 0 & 0 & 0\\
1 & 1 & 1 & 1 & 1
\end{tabular}
$$

Eine \"aquivalente, mathematischere Formulierung f\"ur die XOR-Verkn\"upfung ist das Tripel $(\mathbb{Z}/2\mathbb{Z},+,0)$, wobei $\mathbb{Z}/2\mathbb{Z}$ isomorph ist zu der Menge \{True, False\}. Die Verkn\"upfung $+$ entspricht der Addition auf $\mathbb{Z}/2\mathbb{Z}$ und liefert dadurch auch die Assoziativit\"at, und das Einselement ist hier die $0 \in \mathbb{Z}/2\mathbb{Z}$.
\subsubsection*{v)}
Die Vereinigung von Mengen ist ein Monoid mit dem Tripel $(\mathcal{P}(X), \cup, \emptyset)$, wobei $\mathcal{P}(X)$ die Potenzmenge einer Menge $X$ ist und $\cup$ die Vereinigungsverkn\"upfung darstellt, die nach Definition assoziativ ist. Das Einselement ist hier die leere Menge, da $A \cup \emptyset = A\quad\forall A \subseteq X$. Vereinigt man zwei Elemente aus $\mathcal{P}(X)$, so ergibt sich ein Element, das ebenfalls in $\mathcal{P}(X)$ liegt.


\subsection*{b)}
\subsubsection*{i)}
Welche Form haben die Eingaben der Map- und Reduceschritte?\\
Die Eingabe des Mappers ist $(k,v)$-Tupel mit $k$ als String und $v$ als Integer. Sie bildet ein Monoid bez\"uglich der Konkatenation.\\
Die Eingabe des Reducers ist von der Form $(k,l=[v_1,\dots,v_n])$, also vom Typ $Map<String,List<Integer>>$ und ist ebenfalls ein Monoid bez\"uglich der Konkatenation mit dem Tripel $(Map<String,List<Integer>>,\otimes,\epsilon)$.\\
Welche Form hat die Ausgabe?\\
Die Ausgabe ist vom Typ $Map<String,Integer>$ und hat die Monoidstruktur mit dem Tripel $(Map<String, Integer>, \otimes, \epsilon)$.
\newpage
\subsubsection*{ii)}
Reduce ist ein Monoid-Homomorphismus und kann also als Combiner verwendet werden.\\
\textbf{Denn: } Die Eingabe f\"ur den Reducer sind $(k,l)$-Paare. Reduce summiert alle $l$-Werte f\"ur alle Schl\"ussel auf.\\
Bei einer Aufteilung der Reduce-Eingabe in zwei St\"ucke
($(k, l_{\textit{first}})$- und $(k,l_{\textit{second}})$-Paare, $l=l_{\textit{first}}+l_{\textit{second}}$)
summiert Reduce $l_{\textit{first}}$ und $l_{\textit{second}}$ einzeln und erh\"alt $l$, da die Summe der beiden Teilsummen die ganze Summe ergeben. 


\end{document}
