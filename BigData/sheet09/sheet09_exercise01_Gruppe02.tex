%%%%%%%%%%%%%%%%%%%%%%%%%%%%%%%%%%%%%%%%%
% University/School Laboratory Report
% LaTeX Template
% Version 3.1 (25/3/14)
%
% This template has been downloaded from:
% http://www.LaTeXTemplates.com
%
% Original author:
% Linux and Unix Users Group at Virginia Tech Wiki 
% (https://vtluug.org/wiki/Example_LaTeX_chem_lab_report)
%
% License:
% CC BY-NC-SA 3.0 (http://creativecommons.org/licenses/by-nc-sa/3.0/)
%
%%%%%%%%%%%%%%%%%%%%%%%%%%%%%%%%%%%%%%%%%

%----------------------------------------------------------------------------------------
%	PACKAGES AND DOCUMENT CONFIGURATIONS
%----------------------------------------------------------------------------------------

\documentclass{article}

%\usepackage{algorithm}
%\usepackage[noend]{algpseudocode}


%\usepackage[version=3]{mhchem} % Package for chemical equation typesetting
%\usepackage{siunitx} % Provides the \SI{}{} and \si{} command for typesetting SI units
\usepackage{graphicx} % Required for the inclusion of images
\usepackage{natbib} % Required to change bibliography style to APA
\usepackage{amsmath} % Required for some math elements 

\usepackage{amssymb}

\usepackage{tikz}

\usepackage{listings}
\lstset{numbers=left, numberstyle=\tiny, numbersep=5pt}
\lstset{language=bash}

\setlength\parindent{10pt} % Removes all indentation from paragraphs

\renewcommand{\labelenumi}{\alph{enumi}.} % Make numbering in the enumerate environment by letter rather than number (e.g. section 6)

%\usepackage{times} % Uncomment to use the Times New Roman font

%----------------------------------------------------------------------------------------
%	DOCUMENT INFORMATION
%----------------------------------------------------------------------------------------

%\title{\textsc{Big Data} \\ \textsc{Sheet 05} } % Title
\title{Big Data \"Ubungsblatt 09}
\author{Anton Bulat, Josephine Geiger, Julia Siekiera} % Author name


\date{\today} % Date for the report

\begin{document}

\maketitle % Insert the title, author and date

\section*{Aufgabe 1: Unterschranken an die Replikationsraten}

Gegeben: Map-Reduce-Programm, das alle Pfade der L\"ange $2$ in einem gerichteten Graphen mit $n$ Knoten findet.\\
Eingabe: Liste aller Kanten. Das Tupel $(u,v)$ beschreibt eine Kante von Knoten $u$ nach Knoten $v$.\\
Gesucht: geeignete Unterschranke an die Replikationsrate $r$ mithilfe des Beweisschemas aus der Vorlesung.


\begin{itemize}
\item Schritt 1:\\
Ein Reducer kann mit $q$ Eingaben maximal $\binom q 2 \approx \frac{q^2}{2}$ Ausgabewerte \"uberdecken, also Pfade der L\"ange $2$. Das hei\ss t
$$g(q) = \frac{q^2}{2}.$$

\item Schritt 2:\\
Als Ausgabe werden Tripel von Knoten erwartet, die jeweils einen Pfad der L\"ange $2$ bilden, also zwei Kanten verbinden. In einem gerichteten Graphen mit $n$ Knoten kann es maximal $2\times \binom n 2 = 2\times \frac{n!}{2!(n-2)!} = n\times(n-1)$ Kanten geben.\\
Also ist die Gesamtzahl der vom Problem generierten Ausgaben $$m = \binom{n \times(n-1)} 2 \approx \frac{(n^2-n)^2}{2}.$$

\item Schritt 3:\\
Es gilt $\sum_{i=1}^k g(q_i) \geq m$ mit $g(q_i) = \frac{q_i^2}{2}$ und $m \approx \frac{(n^2-n)^2}{2}$, also\\
$\Rightarrow \sum_{i=1}^k \frac{q_i^2}{2} \geq \frac{(n^2-n)^2}{2}\\ \Leftrightarrow \sum_{i=1}^k q_i^2 \geq (n^2-n)^2$.\\
\item Schritt 4:\\
Da $q \geq q_i$ bleibt die Ungleichung erf\"ullt:
$$\sum_{i=1}^k qq_i \geq (n^2-n)^2\\
\Leftrightarrow q \sum_{i=1}^k q_i \geq (n^2-n)^2\\
\Leftrightarrow \sum_{i=1}^k q_i \geq \frac{(n^2-n)^2}{q}.$$

\item Schritt 5:\\
Die Anzahl der Eingaben entspricht der Anzahl an Kanten, also $n\times(n-1)$. Nach Division durch diesen Wert ergibt sich die untere Schranke f\"ur $r$:
$$ \frac{1}{n\times(n-1)} \sum_{i=1}^k q_i = r \geq \frac{n^2-n}{q}.$$
\end{itemize}

\end{document}
